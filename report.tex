\documentclass[letterpaper,11pt]{article}
\usepackage{graphicx}
\usepackage{subfigure}
\graphicspath{ {images/} }
\usepackage{enumerate}
\usepackage{amssymb}
\usepackage{mathtools}
%\usepackage{minted}
\usepackage{tcolorbox}
\usepackage[margin=1in]{geometry}
\usepackage{caption}
\usepackage{float}
\usepackage{multirow}
\usepackage{hhline}
\usepackage{indentfirst}
\newcommand\numberthis{\addtocounter{equation}{1}\tag{\theequation}}
\renewcommand{\arraystretch}{1.2}
\linespread{1.3}


\begin{document}
%
\title{Decay Heat Model Derivation}
\author{}
\date{}
%
\maketitle
%
Decay heat in nuclear reactors arises from the production of radioactive
actinides and fission products. It is associated with
$\beta$ and $\gamma$ particles released when these unstable nuclides decay.
The decay process follows an exponential profile described by the Bateman
equations.

The Bateman equations for a single-branch decay chain is as follows:
%
\begin{align}
\frac{dN_1 (t)}{dt} &= - \lambda_1 N_1 (t), \\
\frac{dN_i (t)}{dt} &= - \lambda_i N_i (t) + \lambda_{i-1} N_{i-1} (t), \\
\frac{dN_k (t)}{dt} &= \lambda_{k-1} N_{k-1} (t), \\
\intertext{where}
N_i &= \text{ concentration of nuclides $i$} ,\nonumber \\
\lambda_i &= \text{ decay constant of nuclide $i$ [s$^{-1}$].} \nonumber
\end{align}

Nuclide $i$ decays into nuclide $i+1$ with a decay constant of $\lambda_i$,
and nuclide $k$ is a stable nuclide that terminates the decay chain. In
reality, many nuclides often have multiple decay paths and consequently are a
part of complicated decay branches. We can expand on
the transmutation source term to account for decay into nuclide $i$ from
multiple parent nuclides.
%
\begin{align}
\frac{dN_i (t)}{dt} &= - \lambda_i N_i (t) + \sum_j B_{ji} \lambda_j N_j (t)
\\
\intertext{where}
B_{ji} &= \text{ branching ratio for decay from nuclide $j$ to nuclide $i$}
\nonumber
\end{align}

While the Bateman equation is sufficient for calculating the amount of each
radioisotope in irradiated nuclear fuel after discharge, there are two
additional source terms for some radioisotopes during reactor operation:
direct yield from the fission of a fuel nuclide, and transmutation from
another fission product via neutron capture.

The resulting equation for a one neutron group model is:
%
\begin{align}
\frac{dN_i (t)}{dt} &= - \lambda_i N_i (t) + \sum_j [B_{ji} \lambda_j +
\left<\sigma_{ji}\right> \phi (t)] N_j (t) + \sum_l Y_{li} \left<\sigma_{fl}
\right> \phi (t) N_l (t), \\
\intertext{or equivalently}
\frac{dN_i (t)}{dt} &= - \lambda_i N_i (t) + \sum_j [B_{ji} \lambda_j N_j (t)
+ \left<\Sigma_{ji}\right> \phi (t)] + \sum_l Y_{li} \left<\Sigma_{fl}
\right> \phi (t), \\
\intertext{where}
\left< \sigma_{ji} \right> &= \text{ flux-averaged microscopic absorption
cross section for transmutation of nuclide $j$} \nonumber \\
& \text{ to nuclide $i$ [cm$^2$]}, \nonumber \\
\left< \Sigma_{ji} \right> &= \text{ flux-averaged macroscopic absorption
cross section for transmutation of nuclide $j$} \nonumber \\
&\text{ to nuclide $i$ [cm$^{-1}$]}, \nonumber \\
\phi (t) &= \text{ neutron flux at time $t$ [cm$^{-2}$ s$^{-1}$]}, \nonumber \\
Y_{li} &= \text{ direct yield of nuclide $i$ from the fission of nuclide $l$},
\nonumber \\
\left< \sigma_{fl} \right> &= \text{ flux-averaged microscopic fission
cross section of nuclide $l$ [cm$^2$]} \nonumber \\
\left< \Sigma_{fl} \right> &= \text{ flux-averaged macroscopic fission
cross section of nuclide $l$ [cm$^{-1}$]} \nonumber \\
N_l (t) &= \text{ concentration of nuclide $l$ at time $t$ [cm$^{-3}$]}. \nonumber
\end{align}

From here, we multiply throughout by $\lambda_i$ to get the activity since
activity, $A_i = \lambda_i N_i$.
%
\begin{align}
\frac{dA_i (t)}{dt} &= - \lambda_i A_i (t) + \lambda_i \sum_j [B_{ji}
\lambda_j N_j (t) + \left<\Sigma_{ji}\right> \phi (t)] + \lambda_i \sum_l
Y_{li} \left< \Sigma_{fl} \right> \phi (t) \\
\intertext{where}
A_i &=  \text{ (volumetric) decay activity of nuclide }i \text{ [cm$^{-3}$ s$^{-1}$]} \nonumber
\end{align}

Next, we multiply throughout by $\epsilon_i$, the decay energy released
averaged over all possible decay paths.
%
\begin{align}
\epsilon_i \frac{dA_i (t)}{dt} &= - \lambda_i \epsilon_i A_i (t) + \lambda_i
\epsilon_i \sum_j [B_{ji} \lambda_j N_j (t) + \left<\Sigma_{ji}\right> \phi (t)]
+ \lambda_i \epsilon_i \sum_l
Y_{li} \left< \Sigma_{fl} \right> \phi (t) \\
\Rightarrow \frac{d\omega_i (t)}{dt} &= - \lambda_i \omega_i (t) + \lambda_i
\epsilon_i \sum_j [B_{ji}
\lambda_j N_j (t) + \left<\Sigma_{ji}\right> \phi (t)] + \lambda_i \sum_l
\kappa_{li} \left< \Sigma_{fl} \right> \phi (t) \\
\intertext{where}
\epsilon_i &= \text{ average decay heat from the decay of nuclide }i \text{ [J]} \nonumber \\
\kappa_{li} &= \epsilon_i Y_{li} = \text{ average decay heat per fission of nuclide $l$ from the decay of nuclide }i \text{ [J]} \nonumber \\
\omega_i &= \text{ total decay heat power density from nuclide }i \text{ [W
cm$^{-3}$]} \nonumber
\end{align}

Comparing the final equation to the Fiorina model, we observe that
his models ignore the source terms for decay and transmutation from other
isotopes. The same observation can be made for the PyRK model, although in
addition to that, there is an extra $\lambda_i$ term in the remaining source
term.

\pagebreak
%
\section*{PyRK}
%
\begin{align}
	&\text{Power:} & \frac{dp}{dt} &= \frac{\rho - \beta}{\Lambda} p + \sum_j \lambda_{d,j} C_j & \\
	&\text{Decay heat:} & \frac{d\omega_k}{dt} &= \kappa_k p - \lambda_{FP,k} \omega_k & \label{eq:2}
\end{align}

$\omega_k$: Decay heat from FP group $k$

$\kappa_k$: Heat per fission for decay FP group $k$

\section*{Moltres (Current implementation)}
%
\begin{align}
	&\text{Fission power density:} & Q_f &= \sum_g \epsilon_g \Sigma_{f,g} \phi_g & \\
	&\text{Decay heat density:} & \frac{dD_k}{dt} &= f_k Q_f - \lambda_k D_k - u \cdot\nabla D_k + \nabla \cdot D_{diff} \nabla D_k & \label{eq:4} \\
	&\text{Temp:} & \rho c_p \frac{dT}{dt} &= - \rho c_p (\boldsymbol{u} \cdot \nabla T) + \nabla \cdot [(k + k_t) \nabla T] + (1 - \sum_k f_k) Q_f + \sum_k \lambda_k D_k &
\end{align}

$Q$: Total fission heat [W cm$^{-3}$]

$D_k$: Total energy of decay heat precursor concentration in group $k$ [J cm$^{-3}$]

$\epsilon_g$: Total energy per fission (including decay heat) by neutron from energy group $g$ [J]

$f_k$: Fraction of decay heat to total power at steady state

$\rho$: Density of fuel salt [kg cm$^{-3}$]

$k_t$: Turbulent thermal conductivity [W cm$^{-1}$ K$^{-1}$]

\pagebreak

\section*{Fiorina thesis (2013)}
%
\begin{align}
&\text{Fission power density:} & Q &= \sum_g \Sigma_{f,g} \phi_g \epsilon & \\
&\text{Decay heat density:} & \frac{d D_i}{dt} &= f_i \lambda_i Q - \lambda_i D_i
- u \cdot\nabla D_i + \nabla \cdot D_{diff} \nabla D_i & \label{eq:7} \\
&\text{Temp:} & \rho c_p \frac{dT}{dt} &= - \rho c_p (\boldsymbol{u} \cdot \nabla T) + \nabla \cdot [(k + k_t) \nabla T] + (1 - \sum_i f_i) Q + \sum_i D_i &
\end{align}

$Q$: Total fission heat excluding decay heat [W cm$^{-3}$]

$D_i$: Volumetric power generated by $i$th decay heat group [W cm$^{-3}$]

$\epsilon$: Total energy per fission [J]

$f_i$: Fraction of the total fission power associated to $i$th decay heat group

$d$: Density of fuel salt [kg cm$^{-3}$]

$k_t$: Turbulent thermal conductivity [W cm$^{-1}$ K$^{-1}$]

\section*{ANS-5.1-2014}

The 2014 revision introduced an alternative decay heat model where decay heat
is approximated using a sum of 23 exponentials:
%
\begin{align}
DH(t,T) = \sum^{23}_{j=1} \frac{\alpha_{ij}}{\lambda_{ij}} e^{-\lambda_{ij}
t} (1 - e^{-\lambda_{ij} T})
\end{align}

$DH$: Decay heat power from $i$th nuclide [MeV s$^{-1}$ fission$^{-1}$]

$t$: Time after fuel discharge [s]

$T$: Reactor operating time at discharge [s]

$\alpha_{ij}$: Coefficient for $j$th group and $i$th nuclide [?]

$\lambda_{ij}$: Decay constant for $j$th group and $i$ nuclide [s$^{-1}$]

\pagebreak

\section*{Derivation for Fiorina thesis model}

First, we ignore the advection/diffusion of precursors. The equation for the
energy density of decay heat precursors is:
%
\begin{align}
\frac{dC_i}{dt} &= f_i Q - \lambda_i C_i & \text{[J cm}^{-3} \text{s}^{-1} \text{]}
\end{align}
%
where
%
\begin{align*}
C_i &= \text{Energy density of decay heat group }i & &\text{ [J cm}^{-3}] \\
f_i &= \text{Fraction of total fission power associated to decay heat group }i & &\\
Q &= \sum_g \Sigma_{f,g} \phi_g \epsilon & &\text{ [W cm}^{-3}\text{]}\\
\lambda_i &= \text{Decay constant of decay heat group }i & &\text{ [s}^{-1}\text{]}
\end{align*}

Next, we introduce advection and diffusion terms.
%
\begin{align}
\frac{dC_i}{dt} &= f_i Q - \lambda_i C_i - u \cdot\nabla C_i + \nabla \cdot D_{diff} \nabla C_i & \text{[J cm}^{-3} \text{s}^{-1} \text{]}
\end{align}
%
where
%
\begin{align*}
u &= \text{Flow velocity vector} & &\text{ [cm s}^{-1}\text{]} \\
D_{diff} &= \text{Diffusion coefficient} & &\text{ [cm}^2\text{ s}^{-1}\text{]}
\end{align*}

The dimensions of the advection and diffusion terms are consistent with the
rest of the equation because the del operator has units of m$^{-1}$ or
cm$^{-1}$.

Finally, to convert the energy density of the decay heat group to
instantaneous decay heat power density (i.e. concentration to decay activity),
we multiply throughout by $\lambda_i$.
%
\begin{align}
\frac{d \lambda_i C_i}{dt} &= f_i \lambda_i Q - \lambda_i^2 C_i - u \cdot\nabla \lambda_i C_i + \nabla \cdot D_{diff} \nabla \lambda_i C_i & \text{[W cm}^{-3} \text{s}^{-1} \text{]} \\
\Rightarrow \frac{d D_i}{dt} &= f_i \lambda_i Q - \lambda_i D_i - u \cdot\nabla D_i + \nabla \cdot D_{diff} \nabla D_i & \text{[W cm}^{-3} \text{s}^{-1} \text{]}
\end{align}

\end{document}